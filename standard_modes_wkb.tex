\documentclass[11pt]{article}

%% WRY has commented out some unused packages %%
%% If needed, activate these by uncommenting
\usepackage{geometry}                % See geometry.pdf to learn the layout options. There are lots.
%\geometry{letterpaper}                   % ... or a4paper or a5paper or ... 
\geometry{a4paper,left=2.5cm,right=2.5cm,top=2.5cm,bottom=2.5cm}
%\geometry{landscape}                % Activate for rotated page geometry
%\usepackage[parfill]{parskip}    % Activate to begin paragraphs with an empty line rather than an indent

%for figures
%\usepackage{graphicx}

\usepackage{color}
\definecolor{mygreen}{RGB}{28,172,0} % color values Red, Green, Blue
\definecolor{mylilas}{RGB}{170,55,241}
%% for graphics this one is also OK:
\usepackage{epsfig}

%% AMS mathsymbols are enabled with
\usepackage{amssymb,amsmath}

%% more options in enumerate
\usepackage{enumerate}
\usepackage{enumitem}

%% insert code
\usepackage{listings}

\usepackage[utf8]{inputenc}

\usepackage{hyperref}

%% colors
\usepackage{graphicx,xcolor,lipsum}


\usepackage{mathtools}

\usepackage{graphicx}
\newcommand*{\matminus}{%
  \leavevmode
  \hphantom{0}%
  \llap{%
    \settowidth{\dimen0 }{$0$}%
    \resizebox{1.1\dimen0 }{\height}{$-$}%
  }%
}

\title{WKB approximate solutions for standard baroclinic modes}
\author{Cesar B Rocha\thanks{Scripps Institution of Oceanography, University of California, San Diego; \texttt{crocha@ucsd.edu}}}
\date{\today}

\begin{document}

\newcommand{\com}{\, ,}
\newcommand{\per}{\, .}

%% Averages
% Use \bar to over line solo symbols

\newcommand{\av}[1]{\bar{#1}}
\newcommand{\avbg}[1]{\overline{#1}}
\newcommand{\avbgg}[1]{\overline{#1}}

% A nice definition
\newcommand{\defn}{\ensuremath{\stackrel{\mathrm{def}}{=}}}

% equations
\def\beq{\begin{equation}}
\def\eeq{\end{equation}}

% calculus

\newcommand{\p}{\partial}
\newcommand{\ii}{{\rm i}}
\newcommand{\dd}{{\rm d}}
\newcommand{\id}{{\, \rm d}}
\newcommand{\ee}{{\rm e}}
\newcommand{\DD}{{\rm D}}
\newcommand{\wavy}{\text{wavy}}
\newcommand{\qg}{\text{qg}}
\newcommand{\ep}{\epsilon}

\newcommand{\be}{\beta}
\newcommand{\bx}{\boldsymbol{x}}
\newcommand{\by}{\boldsymbol{y}}
\newcommand{\bu}{\boldsymbol{u}}
\newcommand{\bv}{\boldsymbol{v}}


\newcommand{\half}{\tfrac{1}{2}}
\newcommand{\halfrho}{\tfrac{1}{2}}
\newcommand{\rz}{{}}
\newcommand{\bn}{\boldsymbol{\hat n}}
\newcommand{\br}{\boldsymbol{r}}
\newcommand{\bR}{\boldsymbol{R}}
\newcommand{\bA}{\ensuremath {\boldsymbol {A}}}
\newcommand{\bB}{\ensuremath {\boldsymbol {B}}}
\newcommand{\bU}{\ensuremath {\boldsymbol {U}}}
\newcommand{\bE}{\ensuremath {\boldsymbol {E}}}
\newcommand{\bJ}{\ensuremath {\boldsymbol {J}}}
\newcommand{\bXX}{\ensuremath {\boldsymbol {\mathcal{X}}}}
\newcommand{\bFF}{\ensuremath {\boldsymbol {F}}}
\newcommand{\bF}{\ensuremath {\boldsymbol {F}^{\sharp}}}
\newcommand{\bG}{\ensuremath {\boldsymbol G}}
\newcommand{\bSigma}{\ensuremath {\boldsymbol {\Sigma}}}
\newcommand{\bvarphi}{\ensuremath {\boldsymbol {\varphi}}}
\newcommand{\bxi}{\ensuremath {\boldsymbol {\xi}}}
\newcommand{\avbxi}{\overline{\ensuremath {\boldsymbol {\xi}}}}

% math cal

\newcommand{\J}{\mathcal{J}}
\newcommand{\K}{\mathcal{K}}
\newcommand{\cG}{\mathcal{G}}
\newcommand{\cF}{\mathcal{F}}
\newcommand{\cN}{\mathcal{N}}
\newcommand{\cL}{\mathcal{L}}


% san serif for matrices and differential operators
\newcommand{\helmn}{\mathsf{H}_n}
\newcommand{\sA}{\mathsf{A}}
\newcommand{\sB}{\mathsf{B}}
\newcommand{\sG}{\mathsf{G}}
\newcommand{\sI}{\mathsf{I}}
\newcommand{\sJ}{\mathsf{J}}
\newcommand{\sU}{\mathsf{U}}
\newcommand{\sP}{\mathsf{P}}
\newcommand{\sQ}{\mathsf{Q}}
\newcommand{\sR}{\mathsf{R}}
\newcommand{\sL}{\mathsf{L}}
\renewcommand{\sJ}{\mathsf{J}}
\renewcommand{\sI}{\mathsf{I}}
\renewcommand{\L}{\mathsf{L}}
\newcommand{\sM}{\mathsf{M}}
\newcommand{\sT}{\mathsf{T}}
\newcommand{\sGamma}{\mathsf{\Gamma}}
\newcommand{\sbeta}{\mathsf{\beta}}
\newcommand{\sPi}{\mathsf{\Pi}}
\newcommand{\sC}{\mathsf{C}}
\newcommand{\sQy}{\mathsf{Q}}


% angle brackets

\def\la{\langle}
\def\ra{\rangle}
\def\laa{\left \langle}
\def\raa{\right \rangle}


%grads and div's
\newcommand{\bcdot}{\hspace{-0.1em} \boldsymbol{\cdot} \hspace{-0.12em}}
\newcommand{\bnabla}{\boldsymbol{\nabla}}
\newcommand{\bnablaH}{\bnabla_{\! \mathrm{h}}}
\newcommand{\grad}{\bnabla}
\newcommand{\gradH}{\bnablaH}
\newcommand{\curl}{\bnabla \!\times\!}
\newcommand{\diver}{\bnabla \bcdot }
\newcommand{\cross}{\times}
\newcommand{\lap}{\triangle}


%varthetas and thetas
\newcommand{\vth}{\vartheta}
\newcommand{\psii}{\psi^{\mathrm{i}}}
\newcommand{\thb}{\theta^{\mathrm{-}}}
\newcommand{\vthb}{\vartheta^{\mathrm{-}}}
\newcommand{\vthbhat}{{\hat{\vartheta}}^{\mathrm{-}}}
\newcommand{\vThb}{\varTheta^{\mathrm{-}}}
\newcommand{\psib}{\psi^{\mathrm{-}}}
\newcommand{\tht}{\theta^{\mathrm{+}}}
\newcommand{\vtht}{\vartheta^{\mathrm{+}}}
\newcommand{\vththat}{{\hat{\vartheta}}^{\mathrm{+}}}
\newcommand{\vthtbhat}{{\hat{\vartheta}}^{\pm}}
\newcommand{\vTht}{\varTheta^{\mathrm{+}}}
\newcommand{\vthtb}{\vartheta^{\pm}}
\newcommand{\vThtb}{\varTheta^{\pm}}

% N bar
\newcommand{\Nb}{\overline{N}}

%psi's
%Galerking coefficient for psi:
\newcommand{\gpsi}{\breve \psi}
\newcommand{\gphi}{\breve \phi}
\newcommand{\gq}{\breve q}
\newcommand{\gU}{\breve U}
\newcommand{\gQ}{\breve Q}
\newcommand{\gsigma}{\breve \sigma}


\newcommand{\psit}{\psi^{\mathrm{+}}}
\newcommand{\psithat}{{\hat{\psi}}^{\mathrm{+}}}
\newcommand{\psibhat}{{\hat{\psi}}^{\mathrm{-}}}
\newcommand{\psitb}{\psi^{\pm}}
\newcommand{\psitbhat}{{\hat{\psi}}^\pm}
\newcommand{\St}{S^{\mathrm{+}}}
\newcommand{\Sb}{S^{\mathrm{-}}}
\newcommand{\phb}{\phi^{\mathrm{-}}}
\newcommand{\pht}{\phi^{\mathrm{+}}}
\newcommand{\tautb}{\tau^{\pm}}
\newcommand{\sigmatb}{\sigma^{\pm}}


\newcommand{\bur}{\left(\tfrac{f_0}{N}\right)^2}
\newcommand{\ibur}{\left(\tfrac{N}{f_0}\right)^2}
\newcommand{\Nm}{N_{\mathrm{mix}}}
\newcommand{\xim}{\xi_{\mathrm{mix}}}
\newcommand{\hs}{h_*}
\renewcommand{\sp}{\mathsf{p}}
\newcommand{\sr}{\mathsf{r}}
\newcommand{\se}{\mathsf{e}}
\newcommand{\sptb}{\mathsf{p}^\pm}


%nmax is a problem:
%\newcommand{\nmax}{n_{\mathrm{max}}}
\newcommand{\nmax}{\mathrm{N}}

\newcommand{\WKB}{\mathrm{WKB}}
\newcommand{\Lam}{\Lambda}
\newcommand{\tha}{\theta}
\newcommand{\kap}{\kappa}
\newcommand{\bphi}{\boldsymbol{\phi}}
\newcommand{\third}{\tfrac{1}{3}}
\newcommand{\cs}{c^\star}
\newcommand{\nt}{n^{\mathrm{trnc}}}
\newcommand{\sD}{\mathsf{D}}
\newcommand{\sK}{\mathsf{K_2}}
\newcommand{\stheta}{\mathsf{\theta}}
\newcommand{\sphi}{\mathsf{\phi}}
\newcommand{\sq}{\mathsf{q}}
\newcommand{\cosech}{\text{csch}\,}


%%%%%%%%% %%%%
\newcommand{\zp}{z^+}
\newcommand{\zm}{z^-}
\newcommand{\qA}{q^A_{\nmax}}
\newcommand{\psiB}{\psi^B_{\nmax}}
\newcommand{\phiB}{\phi^B_{\nmax}}
\newcommand{\eye}{\boldsymbol{\hat{i}}}
\newcommand{\jay}{\boldsymbol{\hat{j}}}
\newcommand{\kay}{\boldsymbol{\hat{k}}}
\newcommand{\psiG}{\psi^{\mathrm{G}}}
\newcommand{\qG}{q^{\mathrm{G}}}
\newcommand{\uG}{u^{\mathrm{G}}}
\newcommand{\UGN}{U^{\mathrm{G}}_{\nmax}}
\newcommand{\QGN}{Q^{\mathrm{G}}_{\nmax}}
\newcommand{\sumoddn}{\sum_{n = 1, n~ \text{odd}}^{\nmax}}

% bretherton 
\newcommand{\qBr}{q_{\mathrm{Br}}}
\newcommand{\psiBr}{\psi_{\mathrm{Br}}}


\maketitle

In these notes I derive approximate WKB solutions to the standard baroclinic modes of physical oceanography. The elementary textbook example with constant buoyancy frequency is recovered as a special case. 

\section{Pressure modes}

The standard baroclinic modes for pressure, here denoted $\sp_n(z)$,  is defined by the regular Sturm-Liouville eigenproblem

\beq
\label{eigpb}
\sL \sp_n = -\kappa_n^2 \sp_n\com
\eeq
with homogeneous Neumann boundary conditions
\beq
\label{bc}
@z = -h,\,0:\qquad \sp_n' = 0\com
\eeq
and the self-adjoint Linear operator
\beq
\label{strech}
\sL \defn \frac{\dd}{\dd z}\frac{f_0^2}{N^2}\frac{\dd}{\dd z} \per
\eeq
Hence the eigenmodes, $\sp_n$, are orthogonal. The eigennvalue $\kappa_n$ is the deformation wavenumber of the $n$'th mode. It is convenient to normalize the eigenmodes to have unit $L^2$-norm: 
\beq
\label{normalization0}
\frac{1}{H}\int_{-h}^{0}\!\! \sp_n \sp_m \dd z = \delta_{mn}\com
\eeq
where $\delta_{mn}$ is the Kronecker delta. Equation \eqref{eigpb} can be rewritten as
\beq
\label{eigpb_wkb}
\bur \sp_n'' + \left[\bur\right]' \sp_n' + \kappa_n^2 \,\sp_n = 0\per
\eeq
Introducing the following definitions
\beq
\label{notation}
\ep \defn \frac{1}{\kappa_n} \qquad \text{and} \qquad S^2(z) \defn \ibur \per 
\eeq
we have the renotated equation
\beq
\label{dirich_eigpb_wkb_ep}
\ep^2\, \sp_n'' -\ep^2 \left[\log S^2(z)\right]' \sp_n' + S^2(z) \sp_n = 0\per
\eeq
In the WKB spirit we assume that $S^2(z)$ is slowly varying i.e., the buoyancy frequency $N^2(z)$ does not vary very fast. (This assumption may be problematic near the base of the mixed-layer.) We also assume that $\ep$ is small; the accuracy of the WKB solution improves with mode number. We now make the exponential approximation (e.g., Bender and Orszag)
\beq
\sp_n^e \defn \ee^{Q(z)/\ep}\per
\eeq
Hence
\beq
{\sp_n^e}' = \frac{Q'(z)}{\ep}\sp_n^e\com
\eeq
and 
\beq
{\sp_n^e}'' = \left[\left(\frac{Q'(z)}{\ep}\right)^2 + \frac{Q''(z)}{\ep} \right]\sp_n^e\com
\eeq
Next we expand $Q(z)$ in powers of $\ep$
\beq
\label{aseries}
Q(z) = Q_0(z)  + \ep\,Q_1(z) + \ep^2\,Q_2(z) + \mathcal{O}(\ep^3)\per
\eeq
Substituting \eqref{aseries} in \eqref{dirich_eigpb_wkb_ep} we obtain, to lowest order, $\mathcal{O}(\ep^0)$,
\beq
\label{lowest_order_eqn}
Q_0'^2 + S^2(z) = 0\per
\eeq
Thus
\beq
\label{Q0}
Q_0 = \pm \ii \int^z \!\!\!S(\xi) \,\dd \xi  = \pm \ii \tfrac{1}{f_0} \int^z \!\!\!N(\xi) \,\dd \xi \per
\eeq
At next order, $\mathcal{O}(\ep)$, we have
\beq
\label{first_order_eqn}
2\,Q_0'Q_1' + Q_0'' - Q_0'  \left[\log S^2(z)\right]' = 0\per
\eeq
Hence
\beq
\label{Q_1}
Q_1  =   \frac{1}{2} \log S(z) - \frac{1}{2}\log \pm \ii S(z) + \text{const}\,\, \per
\eeq
Notice that the imaginary part in the $\log$ in \eqref{Q_1} just contributes an irrelevant constant. Thus
\beq
Q_1 = \log \sqrt{S(z)} + \text{const} \,\, \per
\eeq
In the most common WKB approximation (a.k.a ``physical optics'') we truncate \eqref{aseries} at $\mathcal{O}(\ep)$. The solution to \eqref{dirich_eigpb_wkb_ep}, consistent with the bottom boundary condition \eqref{bc}, is
\beq
\sp_n^{po} = A_n\, \sqrt{N(z)}\, \cos \left(\frac{\kappa_n}{f_0} \int_{-h}^{z} \!\!\!N(\xi) \dd \xi\right)\com
\eeq
where $A_n$ is a constant. By imposing the boundary condition at $z=0$ \eqref{bc}, we obtain the eigenvalues $\kappa_n$:
\beq
\label{kappan}
\kappa_n = \frac{n \pi \, f_0}{\overline{N}\,h} \com\qquad n=0,1,2,\ldots \com
\eeq
where the mean buoyancy frequency is
\beq
\label{N_avg}
\overline{N} \defn\frac{1}{h} \int_{-h}^0N(\xi)\dd \xi\per
\eeq
The constant $A_n$ is determined by the normalization condition \eqref{normalization0}. We have
\beq
\label{an_eqn}
A_n^2 \, \int_{-h}^{0}\!\! N(z) \cos^2 \left(\frac{\kappa_n}{f_0}\int_{-h}^{z}\!\!\!N(\xi) \dd \xi\right) \dd z = H\com\qquad n\ge 1\per
\eeq
The integral in \eqref{an_eqn} can be evaluated exactly by making the change of variables 
\beq
\eta \defn \frac{\kappa_n}{f_0}\int_{-h}^{z}\!\!\! N(\xi) \dd \xi  \qquad \Rightarrow \qquad \dd\eta = \frac{\kappa_n}{f_0}N(z) \dd z\com
\eeq
and using the expression for the eigenvalues \eqref{kappan}. We obtain 
\beq
A_n = \Big(2/\overline{N} \Big)^{1/2} \com\qquad n\ge 1\per
\eeq
Thus the WKB approximate solution to the standard pressure modes is
\beq
\sp_n^{po} = \left[\frac{2\,N(z)}{\Nb} \right]^{1/2}\!\!\cos\left( \frac{n \pi}{\Nb\,h} \,\,\,\int_{-h}^{z} \!N(\xi) \dd \xi\right)\com\qquad n\ge1\per
\eeq
The amplitude of the baroclinic modes  at the boundaries is independent of the eigenvalue:
\beq
\sp_n^{po}(z=0) = (-1)^{n} \left[\frac{2 N(0)}{\Nb}\right]^{1/2}\com
\eeq
and
\beq
\sp_n^{po}(z=-h) = \left[\frac{2 N(-h)}{\Nb}\right]^{1/2}\per
\eeq

The barotropic mode is not recovered from the WKB solution because $\kappa_0 = 0$. From \eqref{eigpb} we have that with $\kappa_0 = 0$, the barotropic mode is constant, independent of the stratification. With the normalization \eqref{normalization0} we obtain $\sp_0 = 1$.

\subsection*{Constant buoyancy frequency}
With $N = \text{const.}$  the modes are simple sinusoids. That exact result is recovered as a special case of the WKB solution
\beq
\sp_n^{po} = \sqrt{2} \cos\left[n \pi (1+z/h)\right]\per
\eeq

\section{Density modes}
Similarly the baroclinic modes for density, here denoted by $\sr_n$, are defined via the eigenproblem 
\beq
\label{eig_prob_rho}
\sr_n'' = -\kappa_n^2 \ibur \sr_n\com
\eeq
with homogeneous Dirichlet boundary conditions
\beq
\label{rho_bc}
@z = -h,0: \qquad \sr_n = 0\com
\eeq
and normalization
\beq
\label{normalization2}
\frac{1}{h}\int_{-h}^0 \!\!\sr_n \sr_m \dd z = \delta_{mn}\per 
\eeq
Alternatively, we can work on  the approximation from the beginning. The WKB approximate solution to \eqref{eig_prob_rho}-\eqref{rho_bc}, consistent with the bottom boundary conditions \eqref{rho_bc}, is
\beq
\sr_n^{po} = \frac{B_n}{\sqrt{N(z)}} \sin \left(\frac{\kappa_n}{f_0} \int_{-h}^{z} N(\xi)\dd \xi \right) \com
\eeq
The eigenvalues $\kappa_n$ are the same as before \eqref{kappan}. (This should be no surprise because it follows  from the definition of $\sp_n$ and $\sr_n$. Nonetheless,  the verification is a good sanity check.) To find $B_n$ we use the normalization \eqref{normalization2}
\beq
\label{bn_eqn}
B_n^2 \, \int_{-h}^{0}\!\!\frac{1}{N(z)} \sin \left(\frac{\kappa_n}{f_0}\int_{-h}^{z}\!\!\!N(\xi) \dd \xi\right) \dd z = h\com\qquad n\ge 1\per
\eeq
We use a similar trick as above i.e., we change variables with
\beq
\eta \defn \frac{\kappa_n}{N^2(z) f_0}\int_{-h}^{z}\!\!\! N(\xi) \dd \xi  \qquad \Rightarrow \qquad \dd\eta = \frac{\kappa_n}{N(z) f_0} \dd z\com
\eeq
where, in the WKB spirit, we used the fact that $N(z)$ is slowly varying when differentiating the relation above. We obtain
\beq
B_n = \Big(2 N^2(0)/\Nb\Big)^{1/2}\per
\eeq
Thus the WKB approximate solution to the density modes is
\beq
\label{rpo_final}
\sr_n^{po} = \left(\frac{2 N^2(0)}{\Nb N(z)}\right)^{1/2} \sin \left(\frac{n \pi}{\Nb\,h} \int_{-h}^{z} N(\xi)\dd \xi \right)\com\qquad n\ge 1 \per
\eeq
Finally, note that the modes are simply related
\beq
\frac{\dd \sr_n^{po}}{\dd z} = N(0) \underbrace{\frac{n\pi}{\Nb\,h}}_{=\kappa_n/f_0}\,\sp_n^{po}\per
\eeq

\subsection*{Constant buoyancy frequency}
Again we recover the $N = \text{const.}$ special case from \eqref{rpo_final}:
\beq
\sr_n^{po} = \sqrt{2}\sin \left[n\pi(1 + z/h)\right]\per
\eeq

%\section{Probing WKB}
%How accurate is the WKB approximate solution for the standard modes? To answer this questions,
%we first recall that the fundamental assumption in WKB is the slowly varying nature of $N^2(z)$.
%Second, the introduction  perturbation series \eqref{aseries} hinges on the smallness of the parameter
%$\ep$. Hence we expect the accuracy to increase with mode number. Specifically, the physical optics
%approximation truncates the series at first order, so the error decreases as $n^{-2}$.
%
%
%To verify the predictions above for the accuracy of the WKB solution, we consider an example with exponential stratification: $N^2(z) = N_0^2 \ee^{-\alpha z}$,
%where $N_0$ is the stratification frequency at the surface and $\alpha^{-1}$ is the e-folding depth (see figure 1). In this simple example,
%the standard modes can be calculated analytically in terms of Bessel functions (see LaCasce JPO 2012).
%Figure \ref{N2} shows the stratification profiles for two different e-folding scales: $\alpha=2$ and $\alpha=5$. With $\alpha=5$,
%the buoyancy frequency is strongly surface intensified as opposed to the slowly varying $\alpha=2$ profile. Indeed, the relative
%error of the WKB eigenvalue \eqref{kappan} as compared to the exact eigenvalue is more than 10 times smaller for $\alpha=2$
%than for $\alpha=5$ (see figure \ref{rerror}) --- the relative error in the first deformation wavenumber for $\alpha=2$ is spetacularly less than  $1\%$! As expected,
%the error decreases with modes number $n$ as $n^{-2}$. The eigenmodes are also well approximated by the WKB approximation (see figure \ref{eigenstructure}
%for the 4 gravest baroclinic modes with $\alpha=5$).  While there are some differences in amplitude, particularly at depth, the overall structure and zero-crossing depths
%are well captured by the WKB approximation.


%\begin{figure}[!ht]
%\label{N2}
%  \centering
%    \includegraphics[width=0.5\textwidth]{figs/N2.pdf}
%      \caption{The exponential stratification $N^2(z) = N_0^2 \ee^{-\alpha z}$ with $\alpha = 5$ and $\alpha=2$.}
%\end{figure}
%
%\begin{figure}[!ht]
%\label{rerror}
%  \centering
%    \includegraphics[width=0.5\textwidth]{figs/RelativeError.pdf}
%      \caption{The relative error in deformation wavenumber square (the eigenvalue) for exponential stratification $N^2(z) = N_0^2 \ee^{-\alpha z}$ with $\alpha = 5$ and $\alpha=2$. The magnitude of the error decreases with $\alpha$ ---  the buoyancy frequency is more slowly varying with $\alpha=2$.  The error decreases with mode
%      number as $n^{-2}$ ($-2$ slope in log$\times$log space).}
%\end{figure}
%
%\begin{figure}[!ht]
%\label{eigenstructure}
%  \centering
%    \includegraphics[width=0.9\textwidth]{figs/Eigenstructure.pdf}
%      \caption{The pressure modes for exponential stratification $N^2(z) = N_0^2 \ee^{-\alpha z}$ with $\alpha = 5$ calculated exactly (black) and using WKB (blue).}
%\end{figure}
%


\end{document}
